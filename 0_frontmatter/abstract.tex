
% Thesis Abstract -----------------------------------------------------


%\begin{abstractslong}    %uncommenting this line, gives a different abstract heading
\begin{abstracts}        %this creates the heading for the abstract page
Prostate cancer (CaP) is the second most diagnosed cancer in men all over the world.
CaP growth is characterized by two main types of evolution: (i) the slow-growing tumours progress slowly and usually remain confined to the prostate gland; (ii) the fast-growing tumours metastasize from prostate gland to other organs, which might lead to incurable diseases.
Therefore, early diagnosis and risk assessment play major roles in patient treatment and follow-up.
In the last decades, new imaging techniques based on Magnetic Resonance Imaging (MRI) have been developed improving diagnosis.
In practise, diagnosis can be affected by multiple factors such as observer variability and visibility and complexity of the lesions.
In this regard, computer-aided detection and computer-aided diagnosis systems are being designed to help radiologists in their clinical practice.

Our research extensively analyzes the current state-of-the-art in the development of computer-aided diagnosis and detection systems for prostate cancer detection.
Currently, no computer-aided system using all available MRI modalities has been proposed and tested on a common dataset.
Therefore, we propose a new computer-aided system taking advantage of all MRI modalities (i.e., \acs*{t2w}-\acs*{mri}, \acs*{dce}-\acs*{mri}, DW-\acs*{mri}, \acs*{mrsi}).
Particular attention is paid to the normalization of the \acs*{mri} modalities prior to develop our computer-aided system.
This system has been extensively tested on a dataset which has been made publicly available.
\end{abstracts}
%\end{abstractlongs}
%-------------------------------------------------------------------------

\begin{abstractCatalan}

El c\`ancer de pr\`ostata (CaP) \'es el segon c\`ancer m\'es diagnosticat en homes a tot el m\'on.
El creixement del CaP es caracteritza per dos tipus principals d'evoluci\'o: (i) els tumors de creixement lent que progressen lentament i en general romanen confinats en la gl\`andula de la pr\`ostata; (ii) els tumors de creixement r\`apid que desenvolupen met\`astasi de la pr\`ostata a altres \`organs, el que podria conduir a malalties incurables.
Conseq\"uentment, el diagn\`ostic preco\c{c} i l'avaluaci\'o del risc exerceixen un paper important en el tractament del pacient i el seguiment.
En les \'ultimes d\`ecades s'han desenvolupat noves t\`ecniques d'imatge basades en imatge de resson\`ancia magn\`etica (RM, o MRI de l'angl\`es) per millorar el diagn\`ostic.
A la pr\`actica, el diagn\`ostic pot ser afectat per diversos factors com ara la variabilitat de l'observador i la visibilitat i la complexitat de les lesions.
En aquest sentit, s'estan desenvolupant sistemes per a l'ajuda a la detecci\'o i diagn\`ostic per ordinador per ajudar els radi\`olegs en la seva pr\`actica cl\'inica.

La nostra recerca analitza \`ampliament l'estat de l'art en el desenvolupament de sistemes per a l'ajuda a la detecci\'o i diagn\`ostic per ordinador per a la detecci\'o del c\`ancer de pr\`ostata.
En l'actualitat, no hi ha cap sistema d'ajuda al diagn\`ostic que utilitzi totes les modalitats de MRI disponibles i que hagi estat avaluat en un conjunt de dades com\'u.
Per tant, proposem un nou sistema d'ajuda al diagn\`ostic per ordinador aprofitant totes les modalitats de resson\`ancia magn\`etica (\'es a dir \acs*{t2w}-MRI, DCE-MRI, DW-MRI, MRSI).
Com a etapa pr\`evia al desenvolupament del sistema, es presta especial atenci\'o a la normalitzaci\'o de les modalitats de resson\`ancia magn\`etica.
El sistema desenvolupat ha estat avaluat extensivament en un conjunt de dades que s'han posat a disposici\'o p\'ublica.
 
\end{abstractCatalan}

% ---------------------------------------------------------------------- 
\begin{abstractSpanish}

El c\'ancer de pr\'ostata (CaP) es el segundo c\'ancer m\'as diagn\'osticado en hombres en todo el mundo.
El crecimiento del CaP se caracteriza por dos tipos principales de evoluci\'on: (i) los tumores de crecimiento lento que progresan lentamente y por lo general permanecen confinados en la gl\'andula de la pr\'ostata; (ii) los tumores de crecimiento r\'apido que desarrollan met\'astasis de la pr\'ostata a otros \'organos, lo que podr\'ia conducir a enfermedades incurables.
Consecuentemente, el diagn\'ostico precoz y la evaluaci\'on del riesgo desempe\~nan un papel importante en el tratamiento del paciente y el seguimiento. En las \'ultimas d\'ecadas se han desarrollado  nuevas t\'ecnicas de imagen basadas en imagen de resonancia magn\'etica (RM, o MRI del ingl\'es) para mejorar el diagn\'ostico. En la pr\'actica, el diagn\'ostico puede ser afectado por varios factores tales como la variabilidad del observador y la visibilidad y la complejidad de las lesiones.
En este sentido, se est\'an desarrollando sistemas para la ayuda a la detecci\'on y diagn\'ostico por ordenador para ayudar a los radi\'ologos en su pr\'actica cl\'inica.

Nuestra investigaci\'on analiza ampliamente el estado del arte en el desarrollo de sistemas para la ayuda a la detecci\'on y diagn\'ostico por ordenador para la detecci\'on del c\`ancer de pr\'ostata.
En la actualidad, no existe ning\'un sistema de ayuda al diagn\'ostico que utilice todas las modalidades de MRI disponibles y que haya sido evaluado en un conjunto de datos com\'un.
Por lo tanto, proponemos un nuevo sistema de ayuda al diagn\'ostico por ordenador aprovechando todas las modalidades de resonancia magn\'etica (es decir T2W-MRI, DCE-MRI, DW-MRI, MRSI).
Como etapa previa al desarrollo del sistema, se presta especial atenci\'on a la normalizaci\'on de las modalidades de resonancia magn\'etica.
El sistema desarrollado ha sido evaluado extensivamente en un conjunto de datos que se han puesto a disposici\'on p\'ublica.
 
\end{abstractSpanish}

%-------------------------------------------------------------------------
\begin{abstractFrench}

Le cancer de la prostate est le second type de cancer le plus diagnostiqu\'e au monde.
Il est caract\'eris\'e par deux evolutions distinctes : (i) les tumeurs \`a croissances lentes progressent lentement et restent g\'en\'eralement confin\'ees dans la glande prostatique; (ii) les tumeurs \`a croissances rapides se m\'etastasent de la prostate \`a d'autres organes p\'eriph\'eriques, pouvant causer le d\'evelopement de maladies incurables.
C'est pour cela qu'un diagnostic pr\'ecoce et une \'evaluation du risque jouent des r\^oles majeurs dans le traitement et le suivi du patient.
Durant la derni\`ere d\'ec\'enie, de nouvelles m\'ethodes d'imagerie bas\'ees sur l'Imagerie par R\'esonance Magn\'etique (IRM) ont \'et\'e d\'evelop\'ees.
En pratique, le diagnostic clinique peut \^etre affect\'e par de multiples facteurs comme la variabilit\'e entre observateurs et la complexit\'e des l\'esions lues.
Pour ce faire, des syst\`emes de d\'etection et de diagnostic assist\'e par ordinateur (DAO) ont \'et\'e d\'evelop\'es pour aider les radiologistes durant leurs t\^aches cliniques.

Notre recherche analyse extensivement l'\'etat de l'art actuel concernant le d\'evelopement des syst\`emes de DAO pour la d\'etection du cancer de la prostate.
Actuellement, il n'\'existe aucun syst\`eme de DAO utilisant toutes les modalit\'es IRM disponibles et qui plus est, test\'e sur une base de donn\'ees commune.
Par cons\'equent, nous proposons un nouveau syst\`eme de DAO tirant profit de toutes les modalit\'es IRM (i.e., T2W-MRI, DCE-MRI, DW-MRI, MRSI).
Une attention particuli\`ere est port\'ee sur la normalisation de ces donn\'ees multi-param\'etriques avant la conception du syst\`eme de DAO.
De plus, notre syst\`eme de DAO a \'et\'e test\'e sur une base de donn\'ees que nous rendons publique.

\end{abstractFrench}
