% this file is called up by thesis.tex
% content in this file will be fed into the main document
\chapter*{Glossary}

% Glossary entries are defined with the command \nomenclature{1}{2}
% 1 = Entry name, e.g. abbreviation; 2 = Explanation
% You can place all explanations in this separate file or declare them in the middle of the text. Either way they will be collected in the glossary.

% required to print nomenclature name to page header
\markboth{\MakeUppercase{\nomname}}{\MakeUppercase{\nomname}}


% ----------------------- contents from here ------------------------

%\nomenclature{PCa}{Prostate Cancer}
%
%\nomenclature{CZ}{Central Zone}
%\nomenclature{TZ}{Transitional Zone}
%\nomenclature{PZ}{Peripheral Zone}
%
%\nomenclature{PSA}{Prostate Specific Antigen}
%\nomenclature{TRUS}{Transrectal Ultrasound}
%\nomenclature{MRI}{Magnetic Resonance Imaging}
%\nomenclature{MRSI}{Magnetic Resonance Spectroscopy Imaging}

% It will be faster to do the glossary by hand

\begin{multicols}{2} % \begin{multicols}{#columns}[header text][space]  produces a 2 column page for a compact glossary
\begin{footnotesize} % scriptsize(7) < footnotesize(8) < small (9) < normal (10)

\textbf{BPH} - Benign Prostatic Hyperplasia \\

\textbf{PCa} - Prostate Cancer \\

\textbf{CZ} - Central Zone \\

\textbf{TZ} - Transitional Zone \\

\textbf{PZ} - Peripheral Zone \\

\textbf{PSA} - Prostate Specific Antigen \\

\textbf{TRUS} -Transrectal Ultrasound \\

\textbf{MRI} - Magnetic Resonance Imaging \\

\textbf{DWI} - Diffusion-Weighted Imaging \\

\textbf{DCE} - Dynamic Contrast-Enhanced MR Imaging \\

\textbf{MRSI} - Magnetic Resonance Spectroscopy Imaging \\

\label{nom} % target name for links to glossary

\end{footnotesize}
\end{multicols}