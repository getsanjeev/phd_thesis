\subsection{Segmentation}\label{subsec:chp3:img-reg:seg}
The segmentation task consists in delineating the prostate boundaries in the \ac{mri} and is of particular importance for focusing the posterior processing on the organ of interest~\cite{Ghose2012}. 
In this section, only the segmentation methods used in \ac{cad} for \ac{cap} are presented.
An exhaustive review of prostate segmentation methods in \ac{mri} is available in~\cite{Ghose2012}.

%% \setenumerate{listparindent=\parindent,itemsep=10px}
%% \setlist{noitemsep}
%% \begin{enumerate}[leftmargin=*]

%\item[$-$] \textbf{\textit{Manual segmentation:}} 
\paragraph{Manual segmentation}
To highlight the importance of prostate segmentation task in \ac{cad} systems, it is interesting to note the large number of studies which manually segment the prostate organs~\cite{Artan2009,Artan2010,Matulewicz2013,Niaf2011,Niaf2012,Ozer2009,Ozer2010,Puech2009,Vos2008,Vos2008a,trigui2016classification,trigui2017automatic,lehaire2014computer}.
In all the cases, the boundaries of the prostate gland are manually defined in order to limit further processing to only this area.
This approach ensures the right delineation of the organ; nevertheless this procedure is highly time consuming and should be performed by a radiologist.

%\item[$-$] \textbf{\textit{Region-based segmentation:}} 
\paragraph{Region-based segmentation}
\citeauthor{Litjens2012} used a multi-atlas-based segmentation using multi-modal images --- i.e., \ac{t2w}-\ac{mri} and \ac{adc} map --- to segment the prostate with an additional pattern recognition method to differentiate \ac{cg} and \ac{pz}~\cite{Litjens2012}, as proposed in~\cite{Litjens2012a}.
This method consists in three different steps: (i) the registration between each atlas and the multi-modal images, (ii) the atlas selection, and finally (iii) the classification of the prostate voxels into either \ac{cg} or \ac{pz} classes.
Each atlas and the \ac{mri} images are registered two successive registrations: a rigid registration to roughly align the atlases and the \ac{mri} images followed by an elastic registration using a B-spline transformation.
The cost function driving the registration is defined as the weighted sum of the \ac{mi} of both \ac{t2w}-\ac{mri} and \ac{adc} map.
The final atlas is selected using either a majority voting or the \ac{staple} approach~\cite{Warfield2004}.
Subsequently, each voxel within the prostate are classified either as \ac{cg} or \ac{pz} using a \ac{lda} classifier.
Three types of feature are considered to characterized the voxels: (i) anatomy, (ii) intensity, and (iii) texture.
The relative position and the relative distance from the voxel to the border of the prostate encode the anatomical information.
The intensity features consist in the intensity of the voxel in the \ac{adc} coefficient and the T$_2$ map.
The texture features are composed of 5 different features: homogeneity, correlation~\cite{Amadasun1989}, entropy, texture strength~\cite{Li2005a}, and \ac{lbp}~\cite{Ojala1996}.
Finally, the final segmentation is obtained by removing artifacts and smoothing the contour between the zones using the \ac{tps}~\cite{Bookstein1989}.

\citeauthor{Litjens2014} used an almost identical algorithm in~\cite{Litjens2014}, initially proposed for the PROMISE12 challenge~\cite{Litjens2014a}.
Their segmentation method is also based on multi-atlas multi-modal images, but the SIMPLE method~\cite{langerak2010label} is used instead, to combine labels after the registration of the different atlas to obtain the final segmentation.
 
Finally, \citeauthor{rampun2016computerb} recurrently used a method to segment the \ac{pz}~\cite{rampun2015classifying,rampun2015computer,rampun2016computer,rampun2016computerb,rampun2016quantitative}, which is proposed in~\cite{rampun2014detection}.
The \ac{pz} is modelled using a quadratic function driven by the centre of the prostate, the left-most, and the right-most coordinates of the prostate boundaries.

%\item[$-$] \textbf{\textit{Model-based segmentation:}} 
\paragraph{Model-based segmentation}
\citeauthor{Viswanath2009} in~\cite{Viswanath2008a,Viswanath2009} used the \ac{mantra} method as proposed by \citeauthor{Toth2008}.
\ac{mantra}~\cite{Toth2008} is closely related to the \ac{asm} from~\cite{Cootes1995}.
This algorithm consists of two stages: (i) a training stage where a shape and an appearance model are generated and (ii) the actual segmentation based on the learned model. 
For the training stage, a set of landmarks is defined and the shape model is generated as in the original \ac{asm} method~\cite{Cootes1995}.
Then, to model the appearance, a set of $K$ texture images $\{I_1,I_2,\cdots,I_k\}$ based on first and second order statistical texture features is computed.
For a given landmark $l$ with its given neighbourhood $\mathcal{N}(l)$, its feature matrix extracted is expressed as:

\begin{equation}
  f_l = \{ I_1(\mathcal{N}(l)), I_2(\mathcal{N}(l)), \cdots, I_k(\mathcal{N}(l)) \} \ ,
  \label{eq:mantra1}
\end{equation}

\noindent where $I_k(\mathcal{N}(l))$ represents a feature vector obtained by sampling the $k^{\text{th}}$ texture map using the neighbourhood $\mathcal{N}(l)$.
Therefore, multiple landmarks are generated followed by a decomposition using \ac{pca}~\cite{Pearson1901} to learn the appearance variations as in \ac{asm}.

For the segmentation stage, the mean shape learned previously is initialized in the test image.
The same associated texture images as in the training stage are computed.
For each landmark $l$, a neighbourhood of patches are used to sample the texture images and a reconstruction is obtained using the appearance model previously trained.
The new landmark location will be defined as the position where the \ac{mi} is maximal between the reconstructed and original values.
This scheme is performed in a multi-resolution manner as in \cite{Cootes1995}.

Subsequently, \citeauthor{Viswanath2012} in~\cite{Viswanath2012}, used the \ac{weritas} method also proposed in~\citeauthor{Toth2009}.
Similarly to \ac{mantra}, \ac{weritas} is also based on the \ac{asm} formulation~\cite{Toth2009}.
%As with the \ac{mantra} method, \ac{weritas} is based on the \ac{asm} formulation.
%In fact it is very close to the \ac{mantra} itself. 
%The same texture features are used to construct the appearance models, but instead of using \ac{mi} between the landmarks and neighbour patches for adapting the landmark positions, it defines a metric based on the Mahalanobis distance.
They differ in the last stage of the algorithm in which the Mahalanobis distance is used instead of the \ac{mi} metric, to adapt the positions of new landmarks.
In the training stage, the Mahalanobis distance is computed between landmarks and neighbour patches for each of the features.
Subsequently, a new metric is proposed as a linear weighted combination of those Mahalanobis distances which maximizes the correlation with the Euclidean distance between the patches and the true landmarks.
In the segmentation step, this metric is then computed between the initialized landmarks and neighbouring patches in order to update landmark positions, in a similar fashion to other \ac{acm} models.

\citeauthor{Litjens2011} as well as \citeauthor{Vos2012} used an approach proposed in~\cite{Huisman2010} in which the bladder, the prostate, and the rectum are segmented~\cite{Litjens2011,Vos2012}.
The segmentation task is performed as an optimization problem taking 3 parameters into account linked to organ characteristics such as: (i) the shape (i.e., an ellipse), (ii) the location, and (iii) the respective angles between them.
Furthermore, \citeauthor{Litjens2011} used only the \ac{adc} map to encode the appearance~\cite{Litjens2011} whereas \citeauthor{Vos2012} used both \ac{adc} and T$_2$ maps~\cite{Vos2012}.
The cost function, defined as the sum of the deviations, is minimized using a quasi-Newton optimizer.
This rough segmentation is then used inside a Bayesian framework to refine the segmentation.

\citeauthor{giannini2015fully} segmented the prostate with a multi-Otsu thresholding~\cite{otsu1975threshold} in \ac{adc} images~\cite{giannini2015fully}.
Further morphological operations are applied to improve the segmentation.

Only the work of \citeauthor{Tiwari2009} used the \ac{mrsi} modality to segment the prostate~\cite{Tiwari2009}.
The prostate is segmented based on an unsupervised hierarchical spectral clustering.
First, each \ac{mrsi} spectrum is projected into a lower-dimensional space using graph embedding~\cite{Shi2000}.
To proceed, a similarity matrix $W$ is computed using a Gaussian similarity measure from Euclidean distance~\cite{Belkin2001} such that:

\begin{equation}
	W(\mathbf{x},\mathbf{y}) =
	\begin{cases}	
	 	\exp \left( \frac{\| s(\mathbf{x}) - s(\mathbf{y}) \|_2^2}{\sigma^2} \right) \ , & \text{if } \| \mathbf{x} - \mathbf{y} \|_2 < \epsilon \ , \\
	 	0 \ , & \text{if } \| \mathbf{x} - \mathbf{y} \|_2 > \epsilon \ .
	 \end{cases}
	\label{eq:ge1}
\end{equation}

\noindent where $s(\mathbf{x})$ and $s(\mathbf{y})$ are the \ac{mrsi} spectra for the voxels $\mathbf{x}$ and $\mathbf{y}$, respectively, $\sigma$ is the standard deviation of the Gaussian similarity measure, and $\epsilon$ is the parameter to defined an $\epsilon$-neighbourhood.

The projection can be performed as a generalized eigenvector problem such that:
\begin{eqnarray}
  Lu & = & \lambda D u \ , \nonumber \\
  D(\mathbf{x},\mathbf{x}) & = & \sum_{\mathbf{y}} W(\mathbf{x},\mathbf{y}) \ , \label{eq:ge2} \\
  L & = & D-W \ , \nonumber
\end{eqnarray}

\noindent where $D$ is the diagonal weight matrix, $L$ is the Laplacian matrix, $\lambda$ and $u$ represent the eigenvalues and eigenvectors.
Once that the \ac{mrsi} spectra are projected into the lower-dimensional space, a replicate k-means clustering method is used to define 2 clusters.
Subsequently, the data corresponding to the largest cluster is assumed to belong to the non-prostate voxels and thus these voxels are eliminated from the processing.
The full procedure is repeated until the total number of voxels left is inferior to a given threshold experimentally set.

%\end{enumerate}

\subsubsection{Summary}

The segmentation used in \ac{cad} system for the detection of \ac{cap} are summarized in \ac{tab}~\ref{tab:summary-seg}.

\begin{table}
  \caption{Overview of the segmentation methods used in \ac{cad} systems.}
  \scriptsize
  \centering
  \begin{tabular}{l r}
    \toprule
    \textbf{Segmentation methods} & \textbf{References} \\
    \midrule
    \textbf{\ac{mri}-based segmentation:} & \\ \\ [-1.5ex]
    \quad Manual segmentation & \cite{Artan2009,Artan2010,Matulewicz2013,Niaf2011,Niaf2012,Ozer2009,Ozer2010,Puech2009,Vos2008,Vos2008a,Vos2010,Vos2012,trigui2016classification,trigui2017automatic,lehaire2014computer} \\
    \quad Region-based segmentation & \cite{Litjens2012,Litjens2014,rampun2015classifying,rampun2015computer,rampun2016computer,rampun2016computerb,rampun2016quantitative} \\
    \quad Model-based segmentation & \cite{Litjens2011,Viswanath2008a,Viswanath2009,Viswanath2011,Vos2012,giannini2015fully} \\ \\ [-1.5ex]
    \textbf{\ac{mrsi}-based segmentation:} & \\ \\ [-1.5ex]
    \quad Clustering & \cite{Tiwari2009} \\
    \bottomrule
  \end{tabular}
\label{tab:summary-seg}
\end{table}
