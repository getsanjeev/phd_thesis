% this file is called up by thesis.tex
% content in this file will be fed into the main document

\chapter{State of the art in multi-modal magnetic resonance imaging} % top level followed by section, subsection


% ----------------------- contents from here ------------------------

\section{Magnetic resonance imaging techniques}\label{section:stateart:mritechniques}

\subsection{Anatomic T2-weighted magnetic resonance imaging}\label{subsection:stateart:t2}

\subsubsection{Imaging characteristics}

As previously mentioned in Sect. \ref{subsubsection:intro:prostatecancer:diagnosis:mri}, T2-weighted MRI modality allows to clearly differentiate the prostate anatomy \cite{Hricak1987, Hoeks2011}. Indeed, high-intensity-signal of PZ is highly contrasted with low-signal-intensity of CZ or TZ \cite{Hoeks2011} (see Fig.\textbf{Add figure healthy T2}), while PCa is usually characterized by a very low-signal-intensity (\textbf{Add figure PCa T2}). 

Thus, T2-weighted MRI signal could be enough discriminative in order to highlight PCa tissue from healthy tissue in PZ (see Fig.\textbf{Add PCa T2}). However, low-signal-intensity is not always synonym of PCa tissue and can be consecutive to benign abnormalities (e.g., chronic prostatitis, atrophy, scars, post-examination effects, post-treatment side effects and BPH) \cite{Kirkham2006} (see Fig. \textbf{Add figure with BPH}). With an eye to reduce false positive detections due these abnormalities, several studies provided a way to characterized them. Thus, Cruz et al. associated wedge shape and diffuse extensions without mass effect in T2-weighted MRI with highly benign area \cite{Cruz2002}. Haemorrhage due to post-examination (i.e., TRUS biopsy) can be differentiated using T1-weighted MRI \cite{Kaji1998}. Indeed, haemorrhage is characterized by high-signal-intensity on T1-weighted MRI as depicted in Fig. \textbf{Add T1} \cite{Kaji1998}. With the intention of decreasing artefacts, a delay of eight weeks should be observed between a biopsy examination and a MRI examination \cite{Qayyum2004}.

PCa detection and localization in TZ and CZ are more challenging tasks. The signal intensity in these zones are very similar to that representing PCa tissue. However, Akin et al. characterized PCa in CZ and TZ as an homogeneous low-signal-intensity, ill-defined margins, and lack of capsule (see Fig. \textbf{Add PCa in TZ}) \cite{Akin2006}.

\subsubsection{Medical facts}



%
%An exhaustive list of studies using T2-weighted MRI aiming to detect and localize PCa is given in Tab. \ref{table:subsection:stateart:t2}.
%
%
%
%\begin{landscape}
%% In order to avoid any header to fully fit the table
%\pagestyle{plain}
%% Table with the state of the art related to only T2 Weighted image
%\begin{longtable}{>{\centering\arraybackslash}p{3cm} || >{\centering\arraybackslash}m{1.5cm} >{\centering\arraybackslash}m{1.5cm} >{\centering\arraybackslash}m{1.5cm} >{\centering\arraybackslash}m{1.5cm} >{\centering\arraybackslash}m{2cm} >{\centering\arraybackslash}m{3.5cm} || m{6cm}}
%%header table
%\caption{Studies using T2-weighted MRI in order to detect PCa}
%\endhead
%\rowcolor{DarkBlue}
%Study & \# of Subjects & Field Strength (\textit{T}) & Coil Type & Sequence Type & TR/TE (\textit{ms})/(\textit{ms}) & Thick./FoV/Matrix (\textit{mm})/(\textit{mm})/(\textit{pxl}) &  \multicolumn{1}{c}{\cellcolor{DarkBlue}Results}  \\
%%data of the table
%\rowcolor{LightBlue2}
%Rifkin et al. \cite{Rifkin1990}, 1990 & 194 & 1.5 & Body coil & ND & 2500/40-80;60-120;20-80 & 5/24-28/ND & 
%\textbullet \, Detection: $60 \%$ (180/299). \newline
%\textbullet \, Detection for lesions $ < 10 $ mm: $56 \%$ (115/207). \newline
%\textbullet \, Detection for lesions $ < 10 $ mm: $71 \%$ (65/92).
%\\
%\rowcolor{LightBlue1}
%Parivar et al. \cite{Parivar1991}, 1991 & 12 & 1.5 & Endorectal coil & Spin-Echo &  &  & 77 \\
%\rowcolor{LightBlue2}
%Quint et al. \cite{Quint1991}, 1991 & 26 & 1.5 & Body coil & ND & 2500/20-120 & 5/24-28/$128\times256$ & 54 ND ND \\
%\rowcolor{LightBlue1}
%Carter et al. \cite{Carter1991}, 1991 & 53 & 1.5 &  &  &  &  & 77  \\
%\rowcolor{LightBlue2}
%Quinn et al., 1994 & 26 & 1.5 & Body coil & ND & 2500/20-120 & 5/24-28/$128\times256$ & 54 ND ND \\
%\rowcolor{LightBlue1}
%Ellis et al., 1994 & 53 & 1.5 &  &  &  &  & 77  \\
%\rowcolor{LightBlue2}
%Hricak et al., 1994 & 26 & 1.5 & Body coil & ND & 2500/20-120 & 5/24-28/$128\times256$ & 54 ND ND \\
%\rowcolor{LightBlue1}
%Jagger et al., 1996 & 53 & 1.5 &  &  &  &  & 77  \\
%\rowcolor{LightBlue2}
%Ikonen et al., 1998 & 26 & 1.5 & Body coil & ND & 2500/20-120 & 5/24-28/$128\times256$ & 54  \\
%\rowcolor{LightBlue1}
%Scheidler et al., 1999 & 53 & 1.5 &  &  &  &  & 77  \\
%\rowcolor{LightBlue2}
%Akin et al., 2006 & 26 & 1.5 & Body coil & ND & 2500/20-120 & 5/24-28/$128\times256$ & 54 \\
%\rowcolor{LightBlue2}
%Latcham-setty et al., 2007 & 26 & 1.5 & Body coil & ND & 2500/20-120 & 5/24-28/$128\times256$ & 54  \\
%\rowcolor{LightBlue1}
%Futterer et al., 2006 & 53 & 1.5 &  &  &  &  & 77  \\
%\rowcolor{LightBlue1}
%Futterer et al., 2007 & 53 & 1.5 &  &  &  &  & 77  \\
%\rowcolor{LightBlue2}
%Heijmink et al., 2007 & 26 & 1.5 & Body coil & ND & 2500/20-120 & 5/24-28/$128\times256$ & 54 \\
%\rowcolor{LightBlue1}
%Park et al., 2007 & 53 & 1.5 &  &  &  &  & 77 \\
%\rowcolor{LightBlue2}
%Park et al., 2007 & 26 & 1.5 & Body coil & ND & 2500/20-120 & 5/24-28/$128\times256$ & 54 \\
%\rowcolor{LightBlue1}
%Torricelli et al., 2008 & 53 & 1.5 &  &  &  &  & 77 \\
%\rowcolor{LightBlue1}
%Augustin et al., 2009 & 53 & 1.5 &  &  &  &  & 77 \\
%\rowcolor{LightBlue2}
%Porcaro et al., 2009 & 26 & 1.5 & Body coil & ND & 2500/20-120 & 5/24-28/$128\times256$ & 54
%\label{table:subsection:stateart:t2}
%\end{longtable}
%\end{landscape}

\subsection{Dynamic contrast-enhanced magnetic resonance imaging}\label{subsection:stateart:dce}


\subsection{Diffusion weighted imaging}\label{subsection:stateart:dwi}


\subsection{Proton magnetic resonance spectroscopic imaging}\label{subsection:stateart:mrsi}


\section{Fusion of magnetic resonance imaging techniques}

% ---------------------------------------------------------------------------
% ----------------------- end of thesis sub-document ------------------------
% ---------------------------------------------------------------------------