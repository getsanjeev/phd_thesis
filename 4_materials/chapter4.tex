\chapter{Materials}\label{chap:4}

\section{Website development}

\section{Open \acs*{mpmri} data}

\paragraph{\SI{1.5}{\tesla} General Electric scanner}



\paragraph{\SI{3}{\tesla} Siemens scanner}

The \ac{mpmri} data are acquired from a cohort of patients with higher-than-normal level of \ac{psa}.
The acquisition is performed using a \SI{3}{\tesla} whole body \ac{mri} scanner (Siemens Magnetom Trio TIM, Erlangen, Germany) using sequences to obtain \ac{t2w}-\ac{mri}, \ac{dce}-\ac{mri}, \ac{dw}-\ac{mri}, and \ac{mrsi}.
Aside of the \ac{mri} examination, these patients also have undergone a guided-biopsy.
The dataset is composed of a total of 20 patients of which 18 patients have biopsy proven \ac{cap} and 2 patients are ``healthy'' with negative biopsies.
Therefore, 13 patients have a \ac{cap} in the \ac{pz}, 3 patients have \ac{cap} in the \ac{cg}, 2 patients have invasive \ac{cap} in both \ac{pz} and \ac{cg}, and finally 2 patients are considered as ``healthy''.
An experienced radiologist has segmented the prostate organ --- on \ac{t2w}-\ac{mri}, \ac{dce}-\ac{mri}, and \ac{adc}-\ac{mri} --- as well as the prostate zones --- i.e., \ac{pz} and \ac{cg} ---, and \ac{cap} on the \ac{t2w}-\ac{mri}.

A \SI{3}{\mm} slice fat-suppressed \ac{t2w} fast spin-echo sequence (\ac{tr}/\ac{te}/\ac{etl}: \SI{3400}{\ms}/\SI{85}{\ms}/13) is used to acquire images in sagittal and oblique coronal planes, the latter planes being orientated perpendicular or parallel to the prostate \ac{pz} – rectal wall axis.
Three-dimensional \ac{t2w} fast spin-echo (\ac{tr}/\ac{te}/\ac{etl}: \SI{3600}{\ms}/\SI{143}{\ms}/109, slice thickness: \SI{1.25}{\mm}) images are then acquired in an oblique axial plane.
The nominal matrix and \ac{fov} of the 3D \ac{t2w} fast spin-echo images are \SI[product-units=repeat]{320x256}{\milli\metre\squared} and \SI[product-units=repeat]{280x240}{\milli\metre\squared}, respectively, thereby affording sub-millimetric pixel resolution within the imaging plane.

\ac{dce}-\ac{mri} is performed using a fat suppressed 3D T$_1$ VIBE sequence (\ac{tr}/\ac{te}/Flip angle: \SI{3.25}{\ms}/\SI{1.12}{\ms}/\SI{10}{\degree}; Matrix: $256 \times 192$; \ac{fov}: $280 \times 210$ (with \SI{75}{\percent} rectangular \ac{fov}); slab of 16 partitions of \SI{3.5}{\mm} thickness; temporal resolution: \SI{6}{\s}/slab over approximately \SI{5}{\minute}).
A power injector (Medrad, Indianola, USA) is used to provide a bolus injection of Gd-DTPA (Dotarem, Guerbet, Roissy, France) at a dose of \SI{0.2}{\ml} Gd-DTPA/kg of body weight.

\ac{dw}-\ac{mri} images have been acquired using the single-shot spin-echo echo-planar imaging (EPI) technique.
As proposed by \citeauthor{stejskal1965spin}~\cite{stejskal1965spin}, the diffusion-encoding gradients have been applied using a pulsed gradient spin-echo technique resulting in diffusion images acquired at 2 b-values --- i.e., \SI{100}{\second\per\milli\meter\squared} and \SI{800}{\second\per\milli\meter\squared} --- and in the 3 orthogonal directions.
Sequential sampling of the k-space has been used with a \ac{te} of \SI{101}{\ms} and a \ac{tr} of \SI{4200}{\ms} and bandwidth of \SI{1180}{\hertz\per\px}.
Other parameters included a \ac{fov} of \SI{240}{\milli\metre}, an acquisition matrix size of $128 \times 128$ and a slice thickness of \SI{3.5}{\milli\metre}.
The ADC map has been directly generated by the Siemens workstation from the raw data on a pixel-by-pixel basis.

\ac{mrsi} is performed using a water and lipid suppressed double-spin-echo point-resolved spectroscopic (PRESS) sequence optimized for quantification detection of choline and citrate metabolites.
Water and lipid have been suppressed using a dual-band spectral spatial pulse technique.
In order to eliminate signals from adjacent tissues, especially periprostatic lipids and the rectal wall up to 8 outer voxel saturation pulses have been used.
Data sets were acquired as $16 \times 12 \times 16$ --- interpolated to $16 \times 16 \times 16$ phase-encoded spectral arrays, a \ac{te} of \SI{140}{\ms}, a \ac{tr} of \SI{720}{\ms} and \SI{13}{\minute} of acquisition time.
A spectral bandwidth of \SI{1250}{\hertz} has been used with 512 data points.
A combination of an elliptic weighted averaged k-space acquisition scheme 3D filtering of the signal in k-space have been used, the latter in order to reduce intervoxel signal combination.
Shimming has been carried out using the Siemenbens 3D Mapshim routine on a voxel adapted to the volume of the entire prostate gland.
Additional unsuppressed water acquisitions at \ac{te} of \SI{30}{\ms}, \SI{80}{\ms}, and \SI{140}{\ms} of \SI{1.5}{\minute} have also been performed in order to allow quantification with respect to prostate water.
Systematic verification of the global shim --- i.e., over the complete 3D PRESS-selected volume --- revealed line widths at half-height of the water peak of the order of \SIrange{20}{30}{\hertz}, routinely.
The line widths for individual voxels are of the order of \SIrange{8}{12}{\hertz}.
The total examination time, including the time spent positioning the patient, is approximately 45 minutes.

\section{Open source}

\subsection{\texttt{imbalanced-learn} toolbox}

\subsection{\texttt{protoclass} toolbox}

\subsection{Pipeline-data releases}