\section{Summary of the thesis}

This manuscript begins by introducing problematic and societal challenges related to \ac{cap} which give rise to the research motivation of this work.
An overview is given in \acs{chp}\,\ref{chap:2} regarding the \ac{mri} techniques currently used for medical screening.
Therefore, in \acs{chp}\,\ref{chap:3} we focus on reviewing the current state-of-the-art on mono- and \ac{mpmri} \ac{cad} systems for the detection of \ac{cap}.
As a conclusion, the target of this thesis has been fixed to design and investigate a new \ac{mpmri} \ac{cad} system based on all \ac{mri} modalities currently in used in clinical settings.
Additionally, we point out several missing pieces from the given puzzle: (i) no \ac{mpmri} dataset nor \ac{cad} system are currently publicly available, (ii) the knowledge about pre-processing methods in the current \ac{cad} systems is limited, and (iii) the problem of data balancing never has been explored in the past.
Subsequently, we present in \acs{chp}\,\ref{chap:4} the contours of our working materials --- i.e., \ac{mpmri} dataset, source code, communication website --- which we made publicly available along this thesis.
Finally, \acs{chp}\,\ref{chap:5} and \acs{chp}\,\ref{chap:6} present the technical investigations regarding pre-processing and our \ac{mpmri} \ac{cad} system for the detection of \ac{cap}.

\section{Contributions}

The major contributions of this thesis can be summarized as:

\begin{description}
\item[Public \ac{mpmri} dataset] Together with clinicians, we collected, annotated, and publicly made available the first \ac{mpmri} dataset for the detection of \ac{cap}.
\item[Normalization methods] We proposed and extensively investigated two normalization methods for both \ac{t2w}-\ac{mri} and \ac{dce}-\ac{mri} modalities.
\item[\Ac{mpmri} \ac{cad} for \ac{cap} detection] We proposed and extensively investigated a new \ac{mpmri} \ac{cad} for \ac{cap} detection.
This \ac{cad} system uses all current \ac{mri} modalities currently in use in clinical settings.
\end{description}

\section{Avenues for future research}

Although the proposed \ac{mpmri} \ac{cad} system provides satisfactory results, drastic improvements are needed to the system to become in use in clinical environment.
The current \ac{cad} system is taking decision at the voxel level.
Spatial information is, however, an important factor to be included which suggests to move from a voxel-based system to a super-voxel based system.
In \acs{chp}\,\ref{chap:6}, we identified strong image features --- i.e., Gabor filters and phase congruency features --- which should be further investigated.
In the context of the search for large number of quantitative features~\cite{lambin2012radiomics} --- a.k.a. \emph{radiomics} ---, these two types of features have the merit to be further investigated.
Additionally, other strategies to avoid hand-crafted feature detection have to be explored such as deep convolutional neural-networks.
However, the challenge given by the limited number of cases as in many medical applications stresses the need for transfer learning while applying deep-learning~\cite{shin2016deep}.
Subsequently, collecting additional \ac{mpmri} cases would be beneficial to move towards unsupervised learning.