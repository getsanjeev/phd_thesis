\usepackage[table]{xcolor}
%% The amssymb package provides various useful mathematical symbols
\usepackage{amssymb}

%% The amsthm package provides extended theorem environments
\usepackage{amsthm}

%% amsmath for math environment
\usepackage{amsmath}

\DeclareMathOperator*{\argmin}{arg\,min}
\DeclareMathOperator*{\argmax}{arg\,max}
\DeclareMathOperator*{\sign}{sign}

% to break equation
%\usepackage{mathpazo}
%\usepackage{mathptmx}
%\usepackage[mathpazo]{flexisym}
%\usepackage{breqn}

%% For clever reference
%\usepackage{cleveref}

%% color package
\usepackage{color}

%% figure package
\usepackage{epsf,graphicx}
\usepackage{epstopdf}
\usepackage{subfigure}	
\usepackage{transparent}

%% New environment to have some indent inside enumerate environment
\usepackage{enumitem}

%% To create acronym for proper glossary
\usepackage{acro}

%% To number the line in the article
\usepackage{lineno}

%% Environment to include table with notes
\usepackage{array}
\usepackage{threeparttable}
\usepackage{multirow}
%% In order to change size of margin
\usepackage{geometry}
\usepackage{changepage}
\usepackage{lscape}
%% Colorpackage for table
\usepackage{colortbl}
\usepackage{tabularx}
\usepackage{arydshln}

%% To use URL referencing
\usepackage{url}
%\usepackage[hidelinks]{hyperref}

%% In order to draw some graphs
\usepackage{tikz,xifthen}
\usepackage{tikz-qtree}
\usetikzlibrary{decorations.pathmorphing} % noisy shapes
\usetikzlibrary{fit}					% fitting shapes to coordinates
\usetikzlibrary{backgrounds}	% drawing the background after the foreground
\usetikzlibrary{shapes,arrows,shadows}
\usetikzlibrary{calc,decorations.pathreplacing,decorations.markings,positioning}
\usetikzlibrary{snakes,decorations.text,shapes,patterns}
%\usepackage{scalefnt,lmodern,booktabs}

%% Paxkage for cross and tick symbols
\usepackage{pifont}
\newcommand{\cmark}{\large \color{green!60!black!80}\ding{51}}
\newcommand{\mmark}{\large {\color{green!60!black!80}\ding{51}}$^{!}$}
\newcommand{\xmark}{\large \color{red!60!black!80}\ding{55}}
\newcommand{\cmarksmall}{\color{green!60!black!80}\ding{51}}
\newcommand{\mmarksmall}{{\color{green!60!black!80}\ding{51}}$^{!}$}
\newcommand{\xmarksmall}{\color{red!60!black!80}\ding{55}}

\definecolor{autoGuided}{rgb}{ 0.3765    0.7294    0.9412}
\newcommand{\autoGuidedColor}{(light-Blue)}
\definecolor{fullyAuto}{rgb}{ 0.0941    0.3843    0.6627}
\newcommand{\fullyAutoColor}{(dark-blue)}
\definecolor{semiAuto}{rgb}{ 0.0784    0.5059    0.1686}
\newcommand{\semiAutoColor}{(light-green)}
\definecolor{fullyGuided}{rgb}{ 0.4275    0.6902    0.3176}
\newcommand{\fullyGuidedColor}{(dark-green)}
